\documentclass[11pt]{article}
\usepackage[utf8]{inputenc}
\usepackage{polski}
\usepackage{graphicx}
\usepackage{array}
\usepackage{paralist}
\usepackage{verbatim}
\usepackage{subfig}
\usepackage{amsmath}
\usepackage{float}
\usepackage{amsthm}
\usepackage{amssymb}
\usepackage{pdfpages}
\usepackage{amsfonts}
\usepackage{tikz}
\usepackage[linguistics]{forest}
\usetikzlibrary{shapes,backgrounds}
\usepackage[margin=1in]{geometry}
\setlength\parindent{0pt}
\theoremstyle{definition}
\newtheorem{zadanie}{Zadanie}
\renewcommand*{\proofname}{Rozwiązanie}
\maxdeadcycles=1000
\extrafloats{1000}
\title{Algebra i analiza matematyczna}
\author{Igor Nowicki}
\begin{document}
\maketitle
\tableofcontents

\section{Zadania z egzaminów}

\subsection{Funkcjonały dwuliniowe}

\begin{zadanie}
    Dla jakich parametrów $m\in\mathbb R$ funkcjonał dwuliniowy

    $$f((x_1,x_2), (y_1,y_2)) = x_1y_1+mx_1y_2+mx_2y_1+(3-2m)x_2y_2$$

    jest iloczynem skalarnym w $\mathbb R^2$? Dla $m=-1$ wyznaczyć bazę $\mathbb R^2$, w której $f$ ma macierz diagonalną.
\end{zadanie}

\begin{proof}
Musimy wyznaczyć wartości parametru, dla którego funkcjonał dwuliniowy staje się iloczynem skalarnym. Z założenia, iloczyny skalarne mają szereg wymogów - to, by były symetryczne (tożsamość ze swoją transpozycją) oraz by były dodatnio określone (ich forma diagonalna musi zawierać same dodatnie liczby). Ponieważ obydwie forma jest symetryczna, pozostaje warunek dodatniej określoności. Warunkiem koniecznym (ale nie wystarczającym) dodatniej określoności jest by wyznacznik macierzy był dodatni. Dodatkowo, wszystkie wartości własne muszę być większe od zera.

Zatem:

$$\det\begin{bmatrix}
    1&m\\
    m&(3-2m)
\end{bmatrix} > 0$$

Następnie - potrzebujemy wyznaczyć taką bazę, dla której 

$$\big[f\big] = \begin{bmatrix}
    1&-1\\
    -1&5
\end{bmatrix}$$

ma macierz diagonalną. Polega to na szukaniu takiej macierzy $P$ o wyznaczniku $\det P=1$, dla której spełniony jest warunek:

$$P^T\big[f\big]P = \text{diag}(\lambda_1, \lambda_2),$$

gdzie $\lambda_1, \lambda_2$ są \textbf{wartościami własnymi} macierzy. Otóż, wartości własne $\lambda$ to rozwiązania równania:

$$\det\begin{bmatrix}
    1-\lambda&-1\\
    -1&5-\lambda
\end{bmatrix} = 0,$$

co przekształca się do równania:

$$(1-\lambda)(5-\lambda)-1=0,$$

o rozwiązaniach $\lambda_1 = 3-\sqrt{5}$ oraz $\lambda_2 = 3+\sqrt{5}$.

Dla tak znalezionych wartości należy teraz znaleźć wektory będące \textbf{jądrem} odpowiadających im macierzy. Dla $\lambda_1=3-\sqrt{5}$ mamy:

$$\ker\begin{bmatrix}-2+\sqrt5 & -1\\-1&2+\sqrt5\end{bmatrix} = \begin{pmatrix}1\\-2+\sqrt{5}\end{pmatrix} = \mathbf v_1,$$

oraz, dla drugiej wartości własnej $\lambda_2 = 3+\sqrt5$:

$$\ker\begin{bmatrix}-2-\sqrt5 & -1\\-1&2-\sqrt5\end{bmatrix} = \begin{pmatrix}2-\sqrt{5}\\1\end{pmatrix} = \mathbf v_2,$$

(można istotnie zauważyć, że $v_1\circ v_2 = 0$, co oznacza że są wzajemnie ortogonalne, to znaczy prostopadłe). Mając już wektory własne, jesteśmy w stanie wyznaczyć macierz zmiany bazy $P_{fe}$. Będą to po prostu wektory własne ustawione obok siebie kolumnami. Narzucamy jeszcze warunek normalizacji, tj. nasza macierz powinna być podzielona przez taką liczbę, by jej wyznacznik był równy 1. W tym wypadku będzie to pierwiastek z wyznacznika:

$$P_{fe} = \frac1{\sqrt{10-4\sqrt{5}}}\begin{bmatrix}1&2-\sqrt5\\-2+\sqrt5&1\end{bmatrix}.$$

Odwrotność tej macierzy to będzie nasza poszukiwana macierz zmiany bazy $P_{ef}$. Szczęśliwie się składa, że dla bazy macierzy symetrycznej będzie to po prostu transponowana macierz $P_{fe}:$

$$P_{ef} = P^T_{fe} = \frac1{\sqrt{10-4\sqrt{5}}}\begin{bmatrix}1&-2+\sqrt5\\2-\sqrt5&1\end{bmatrix}.$$

Możemy sprawdzić, że istotnie ta macierz diagonalizuje naszą formę dwuliniową:

$$\frac1{\sqrt{10-4\sqrt{5}}}\begin{bmatrix}1&2-\sqrt5\\-2+\sqrt5&1\end{bmatrix}\circ\begin{bmatrix}
    1&-1\\
    -1&5
\end{bmatrix}\circ\frac1{\sqrt{10-4\sqrt{5}}}\begin{bmatrix}1&-2+\sqrt5\\2-\sqrt5&1\end{bmatrix}$$

co przekształca się do:


$$\frac1{10-4\sqrt{5}}\begin{bmatrix}1&2-\sqrt5\\-2+\sqrt5&1\end{bmatrix}\circ\begin{bmatrix}
    1&-1\\
    -1&5
\end{bmatrix}\circ\begin{bmatrix}1&-2+\sqrt5\\2-\sqrt5&1\end{bmatrix} = \begin{bmatrix}3 - \sqrt5& 0\\ 0& 3 + \sqrt5\end{bmatrix}$$

\end{proof}

\begin{zadanie}
Dla jakich parametrów $k\in\mathbb R$ funkcjonał dwuliniowy $f:\mathbb R^2\times\mathbb R^2\to\mathbb R$ dany wzorem:

$$f((x_1,x_2)(y_1,y_2))=(k-3)x_1y_1+2x_1y_2+2x_2y_1+kx_2y_2$$

jest iloczynem skalarnym? Dla $k=1$ wyznaczyć macierz $f$ w bazie $b_1=(1,-1), b_2=(0,2)$.
\end{zadanie}

Ponownie, zadanie wyznaczenia parametru sprowadza się do postawienia warunku wyznacznika większego niż zero. Jeśli chcemy uzyskać macierz przejścia do konkretnej bazy, to pamiętamy, że macierz przejścia z bazy $b$ do bazy kanonicznej do postawione kolumnowo obok siebie wektory $b_i$.

$$P = \begin{bmatrix}1&0\\-1&2\end{bmatrix}$$

% Musimy znaleźć macierz odwrotną. Możemy to zrobić przez algorytm odwracania macierzy:

% $$B_{eb} = B_{be}^{-1} = \begin{bmatrix}1&0\\-1&2\end{bmatrix}^{-1}= \begin{bmatrix}1&0\\\frac12&\frac12\end{bmatrix}$$

% Potrzebujemy znormalizowanej macierzy zmiany bazy - będzie ona miała postać:

% $$P = \frac1{\sqrt{|B_{eb}|}}B_{eb} =\begin{bmatrix}\sqrt{2}&0\\\frac{\sqrt{2}}2&\frac{\sqrt{2}}2\end{bmatrix}$$

Aby uzyskać macierz funkcjonału dwuliniowego w nowej bazie, dokonujemy następujących przekształceń:

$$\big[f\big]_b =\frac1{\det P} P^T\circ\big[f\big]_e\circ P$$

\begin{zadanie}
    W przestrzeni $\mathbb R^3$ z iloczynem skalarnym

    $$f((x_1,x_2,x_3)(y_1,y_2,y_3)) = x_1y_1-x_1y_2-x_2y_1+2x_2y_2+x_3y_3$$

    wyznaczyć bazę ortogonalną $\mathbb R^3$, która zawiera bazę podprzestrzeni

    $$V=\{(x_1,x_2,x_3)\in\mathbb R^3:x_1+x_2+x_3=0\}.$$
\end{zadanie}


% \begin{proof}
%     Przedstawiony powyżej funkcjonał liniowy można opisać macierzą:

%     $$\mathcal M_f = \begin{bmatrix}
%             1 & m    \\
%             m & 3-2m
%         \end{bmatrix}.$$

%     Aby forma liniowa mogła być iloczynem skalarnym, musi być dodatnio określona, tj. jej wyznacznik musi być większy od zera. Zatem:

%     $$\det\mathcal M_f = \begin{vmatrix}
%             1 & m    \\
%             m & 3-2m
%         \end{vmatrix} = 1\cdot(3-2m) - m\cdot m > 0,$$

%     co sprowadza się do nierówności:
%     \begin{align*}
%         m^2+2m-3   & < 0, \\
%         (m+3)(m-1) & < 0.
%     \end{align*}

%     która ma rozwiązania dla $m\in (-3, 1)$.

%     Dalej - chcemy wyznaczyć bazę dla przypadku kiedy $m = -1$, to znaczy, kiedy macierz przyjmuje formę:

%     $$\mathcal M = \begin{bmatrix}
%             1  & -1 \\
%             -1 & 5
%         \end{bmatrix}.$$

%     Bazą jest minimalny układ wektorów które opisują nam przestrzeń w której się poruszamy - ponieważ funkcjonał liniowy działa na przestrzeni dwuwymiarowej,
%     ożemy o bazie myśleć jako o dwóch wektorach:
%     $x_1 = (1,0)$ oraz $x_2=(0,1)$. Poszukujemy takiej nowej kombinacji wektorów, dla której nasza forma liniowa przyjmie postać diagonalną, tj. niezerowe wartości będą jedynie na przekątnej. Sprowadza się to do znalezienia takiej macierzy $B$, dla której:

%     $$B\cdot M\cdot B^{-1} =  \begin{bmatrix}
%             \lambda_1 & 0         \\
%             0         & \lambda_2
%         \end{bmatrix}$$

%     Wartości $\lambda_1$, $\lambda_2$ możemy znaleźć przez szukanie rozwiązań równania:

%     $$\det \begin{bmatrix}
%             1-\lambda & -1        \\
%             -1        & 5-\lambda
%         \end{bmatrix} = 0,$$

%     które przekształca się do równania kwadratowego $(1-\lambda)(5-\lambda) -1 = 0.$ Jego rozwiązaniami są $\lambda_1 = 3-\sqrt5$ oraz $\lambda_2=3+\sqrt5$.

%     Bazą diagonalizującą będzie układ takich wektorów, dla których spełnione będą równania:

%     \begin{align*}
%         \begin{bmatrix}
%             1-\lambda_1 & -1          \\
%             -1          & 5-\lambda_1
%         \end{bmatrix} \cdot v_1  & = \begin{bmatrix}0\\0\end{bmatrix},  \\
%         \begin{bmatrix}
%             1-\lambda_2 & -1          \\
%             -1          & 5-\lambda_2
%         \end{bmatrix} \cdot v_2 & = \begin{bmatrix}0\\0\end{bmatrix},
%     \end{align*}

%     Rozwiązaniem będą wektory $v_1 = (2+\sqrt5, 1)$ oraz $v_2 = (2-\sqrt5, 1)$.

%     Algorytm postępowania do znalezienie parametru $m$ dla którego funkcjonał liniowy jest iloczynem skalarnym:
%     \begin{enumerate}
%         \item Utworzyć macierz na podstawie funkcjonału,
%         \item sprawdzić w jakich warunkach wyznacznik macierzy przyjmuje wartość powyżej zera.
%     \end{enumerate}

%     Algorytm postępowania do znalezienia bazy diagonalizującej:

%     \begin{enumerate}
%         \item Utworzyć macierz funkcjonału $f$,
%         \item znaleźć wartości własne macierzy,
%         \item znaleźć wektory własne odpowiadające wartościom własnym macierzy.
%     \end{enumerate}
% \end{proof}

Podstawowe informacje:

Jeśli mamy funkcjonał liniowy wyrażony jako $F((x_1,\dots,x_n)(y_1,\dots,y_n)) = \sum_{i,j=1}^na_{ij}x_iy_j$, to można go przedstawić za pomocą macierzy:

$$\Big[F\Big] = \begin{bmatrix}
    a_{11} & \dots & a_{1n}\\
    \vdots & \ddots & \vdots\\
    a_{n1} & \dots & a_{nn}
    \end{bmatrix}$$

    Jeśli chcemy narzucić warunek by funkcjonał liniowy był jednocześnie iloczynem skalarnym, to forma musi być symetryczna, oraz jej wyznacznik musi być większy od zera. Dla macierzy $2\times2$ z parametrami sprowadza się to do rozwiązywania równania kwadratowego.

    Jeśli chcemy znaleźć bazę diagonalizującą, to musimy znaleźć takie wektory, które przemnożone przez macierz funkcjonału liniowego dają same siebie przemnożone przez liczbę - musimy znaleźć \textbf{wektory własne}.
\subsection{Znajdywanie jądra}

\begin{zadanie}
Wyznaczyć bazę ortogonalną podprzestrzeni $V\in \mathbb R^4$ o równaniu $x_1-2x_2+x_3-x_4=0$.
\end{zadanie}

\begin{zadanie}
    Wyznaczyć bazę ortogonalną podprzestrzeni $V\subset \mathbb R^4$ o równaniu $x_1-2x_2+x_3-x_4 = 0$.
\end{zadanie}
\begin{proof}
    Pojedyncze równanie dla przestrzeni czterowymiarowej ogranicza nam podprzestrzeń do trzech wymiarów. Znalezienie bazy ortogonalnej oznacza znalezienie takich trzech wektorów które:
    \begin{enumerate}
        \item spełniają podane równanie (należą do \textit{jądra} przekształcenia liniowego),
        \item są wzajemnie prostopadłe, tj. $\mathbf v_i\cdot\mathbf v_j = 0$ dla $i\neq j$.
    \end{enumerate}

    Pierwszy warunek pozwala nam na znalezienie całej masy wektorów - przykładem mogą być $v_1=(2,1,0,0), v_2 = (0,1,1,0), v_3=(0,0,1,1)$. Jedynym problemem jest fakt, że nie są to wektory wzajemnie prostopadłe - ich wzajemne iloczyny skalarne nie są równe zeru. Z pomocą może przyjść \textbf{ortogonalizacja Grama-Schmitda} - algorytm przekształcania dowolnego zestawu wektorów liniowo niezależnych do bazy ortogonalnej.

    Dla zestawu trzech wektorów procedura przebiega następująco:

    \begin{align*}
        \mathbf u_1 & = \mathbf v_1,                                                                                                                                                              \\
        \mathbf u_2 & = \mathbf v_2 - \frac{\mathbf v_2\cdot \mathbf u_1}{\mathbf u_1\cdot \mathbf u_1}\mathbf u_1,                                                                               \\
        \mathbf u_3 & = \mathbf v_3 - \frac{\mathbf v_3\cdot \mathbf u_2}{\mathbf u_2\cdot \mathbf u_2}\mathbf u_2- \frac{\mathbf v_3\cdot \mathbf u_1}{\mathbf u_1\cdot \mathbf u_1}\mathbf u_1.
    \end{align*}
\end{proof}

\subsection{Rzuty ortogonalne}

\begin{zadanie}
    Wyznaczyć rzut ortogonalny wektora $\mathbf u=(1,-2,3)$ na prostą $L\subset \mathbb R^3$ o równaniach:

    $$x_1-x_3 = x_1+2x_2+x_3 = 0.$$

    Obliczyć odległość wektora $\mathbf u$ od tej prostej.
\end{zadanie}
\begin{proof}
    Algorytm postępowania:
    \begin{enumerate}
        \item Znaleźć jądro przekształcenia liniowego - to będzie nasza baza podprzestrzeni liniowej, w tym wypadku prostej.
        \item Przeprowadzić rzut ortogonalny wektora $\mathbf u$ na podprzestrzeń rozpinaną przez wektor $\mathbf v$ - będzie to opisane wzorem $P(\mathbf u) = \frac{\mathbf v\cdot \mathbf u}{\mathbf v\cdot \mathbf v}\mathbf v$.
        \item Aby obliczyć odległość wektora od prostej, należy od wejściowego wektora $u$ odjąć wektor uzyskany przez rzucenie na podprzestrzeń, a następnie wyliczyć jego długość.
    \end{enumerate}
\end{proof}

\subsection{Wypukłości zbiorów}

\begin{zadanie}
Zbadać wypukłość zbioru $W\in\mathbb R^2$ danego warunkami

$$x^2-2x+2y^2\leq 1\wedge y\leq\min\{\ln(x+e),\sqrt{1-x}\}\wedge -e\leq x\leq 1.$$
\end{zadanie}

\begin{zadanie}
    Zbadać wypukłość zbioru $W\subset \mathbb R^2$ danego warunkami:

    $$x^4+y^2\leq 1 \wedge y\geq \max\{e^{x^2}-1, -\ln(x+1)\}\wedge x>-1.$$
\end{zadanie}

% \begin{proof}
% Przecięcie zbiorów wypukłych jest zbiorem wypukłym. Możemy rozbić powyższe warunki na następujące:

% \begin{itemize}
% \item $x^4+y^2\leq 1$
% \item $y\geq e^{x^2}-1$ dla $x>0$
% \item $y\geq -\ln(x+1)$ dla $x\leq0$.
% \end{itemize}

% (warunek $x>-1$ jest spełniany automatycznie przy konieczności pierwszego warunku).
% \end{proof}

\subsection{Równania różniczkowe}

\begin{zadanie}
Wyznaczyć rozwiązanie równania różniczkowego

$$y' +2xy=\cos x e^{x^2}y^2$$

spełniające warunek początkowy $y(0)=1$.
\end{zadanie}
\begin{proof}
    Należy użyć funkcji pomocniczej, bądź dokonać podstawienia $z=ye^{x^2}$.
\end{proof}

\begin{zadanie}
Wyznaczyć rozwiązanie ogólne równania różniczkowego
$$y''-6y'+9y = xe^x.$$
\end{zadanie}

\begin{zadanie}
Wyznaczyć rozwiązanie równania różniczkowego:

$$y' - \frac yx = \frac{y^2}{\sqrt{x^2+1}}$$

spełniające warunek $y(1)=1$.
\end{zadanie}
\begin{proof}
Należy dokonać podstawienia:

$$y = vx,$$

rozwiązać równanie dla $v(x)$ a następnie podstawić $y$ z powrotem.
\end{proof}

\begin{zadanie}
Podać rozwiązanie ogólne równania różniczkowego:

$$y'' - 4y' + 4y = xe^{-x}.$$
\end{zadanie}

\begin{proof}
Należy wykonać podstawienie $z = y' - 2y$.
\end{proof}

\begin{zadanie}
Wyznaczyć rozwiązanie ogólne układu równań różniczkowych:

\begin{align*}
    \begin{cases} y'&= -y+3z-1,\\
z'&=3y-z-5.
    \end{cases}
\end{align*}
\end{zadanie}
\begin{proof}
    Należy znaleźć wektory własne przekształcenia liniowego a następnie dokonać podstawienia.
\end{proof}

\end{document}
