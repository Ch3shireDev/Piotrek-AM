\documentclass[11pt]{article}
\usepackage[utf8]{inputenc}
\usepackage{polski}
\usepackage{graphicx}
\usepackage{array}
\usepackage{paralist}
\usepackage{verbatim}
\usepackage{subfig}
\usepackage{amsmath}
\usepackage{float}
\usepackage{amsthm}
\usepackage{amssymb}
\usepackage{pdfpages}
\usepackage{amsfonts}
\usepackage{tikz}
\usepackage[linguistics]{forest}
\usetikzlibrary{shapes,backgrounds}
\usepackage[margin=1in]{geometry}
\setlength\parindent{0pt}
\theoremstyle{definition}
\newtheorem{zadanie}{Zadanie}
\renewcommand*{\proofname}{Rozwiązanie}
\maxdeadcycles=1000
\extrafloats{1000}
\title{Algebra i analiza matematyczna}
\author{Igor Nowicki}
\begin{document}
\maketitle
\tableofcontents

\section{Zadania z egzaminów}

\begin{zadanie}
    Dla jakich parametrów $m\in\mathbb R$ funkcjonał liniowy

    $$f((x_1,x_2), (y_1,y_2)) = x_1y_1+mx_1y_2+mx_2y_1+(3-2m)x_2y_2$$

    jest iloczynem skalarnym w $\mathbb R^2$? Dla $m=-1$ wyznaczyć bazę $\mathbb R^2$, w której $f$ ma macierz diagonalną.
\end{zadanie}
\begin{proof}
    Przedstawiony powyżej funkcjonał liniowy można opisać macierzą:

    $$\mathcal M_f = \begin{bmatrix}
            1 & m    \\
            m & 3-2m
        \end{bmatrix}.$$

    Aby forma liniowa mogła być iloczynem skalarnym, musi być dodatnio określona, tj. jej wyznacznik musi być większy od zera. Zatem:

    $$\det\mathcal M_f = \begin{vmatrix}
            1 & m    \\
            m & 3-2m
        \end{vmatrix} = 1\cdot(3-2m) - m\cdot m > 0,$$

    co sprowadza się do nierówności:
    \begin{align*}
        m^2+2m-3   & < 0, \\
        (m+3)(m-1) & < 0.
    \end{align*}

    która ma rozwiązania dla $m\in (-3, 1)$.

    Dalej - chcemy wyznaczyć bazę dla przypadku kiedy $m = -1$, to znaczy, kiedy macierz przyjmuje formę:

    $$\mathcal M = \begin{bmatrix}
            1  & -1 \\
            -1 & 5
        \end{bmatrix}.$$

    Bazą jest minimalny układ wektorów które opisują nam przestrzeń w której się poruszamy - ponieważ funkcjonał liniowy działa na przestrzeni dwuwymiarowej,
    ożemy o bazie myśleć jako o dwóch wektorach:
    $x_1 = (1,0)$ oraz $x_2=(0,1)$. Poszukujemy takiej nowej kombinacji wektorów, dla której nasza forma liniowa przyjmie postać diagonalną, tj. niezerowe wartości będą jedynie na przekątnej. Sprowadza się to do znalezienia takiej macierzy $B$, dla której:

    $$B\cdot M\cdot B^{-1} =  \begin{bmatrix}
            \lambda_1 & 0         \\
            0         & \lambda_2
        \end{bmatrix}$$

    Wartości $\lambda_1$, $\lambda_2$ możemy znaleźć przez szukanie rozwiązań równania:

    $$\det \begin{bmatrix}
            1-\lambda & -1        \\
            -1        & 5-\lambda
        \end{bmatrix} = 0,$$

    które przekształca się do równania kwadratowego $(1-\lambda)(5-\lambda) -1 = 0.$ Jego rozwiązaniami są $\lambda_1 = 3-\sqrt5$ oraz $\lambda_2=3+\sqrt5$.

    Bazą diagonalizującą będzie układ takich wektorów, dla których spełnione będą równania:

    \begin{align*}
        \begin{bmatrix}
            1-\lambda_1 & -1          \\
            -1          & 5-\lambda_1
        \end{bmatrix} \cdot v_1  & = \begin{bmatrix}0\\0\end{bmatrix},  \\
        \begin{bmatrix}
            1-\lambda_2 & -1          \\
            -1          & 5-\lambda_2
        \end{bmatrix} \cdot v_2 & = \begin{bmatrix}0\\0\end{bmatrix},
    \end{align*}

    Rozwiązaniem będą wektory $v_1 = (2+\sqrt5, 1)$ oraz $v_2 = (2-\sqrt5, 1)$.

    Algorytm postępowania do znalezienie parametru $m$ dla którego funkcjonał liniowy jest iloczynem skalarnym:
    \begin{enumerate}
        \item Utworzyć macierz na podstawie funkcjonału,
        \item sprawdzić w jakich warunkach wyznacznik macierzy przyjmuje wartość powyżej zera.
    \end{enumerate}

    Algorytm postępowania do znalezienia bazy diagonalizującej:

    \begin{enumerate}
        \item Utworzyć macierz funkcjonału $f$,
        \item znaleźć wartości własne macierzy,
        \item znaleźć wektory własne odpowiadające wartościom własnym macierzy.
    \end{enumerate}

\end{proof}

\begin{zadanie}
    Wyznaczyć bazę ortogonalną podprzestrzeni $V\subset \mathbb R^4$ o równaniu $x_1-2x_2+x_3-x_4 = 0$.
\end{zadanie}
\begin{proof}
    Pojedyncze równanie dla przestrzeni czterowymiarowej ogranicza nam podprzestrzeń do trzech wymiarów. Znalezienie bazy ortogonalnej oznacza znalezienie takich trzech wektorów które:
    \begin{enumerate}
        \item spełniają podane równanie (należą do \textit{jądra} przekształcenia liniowego),
        \item są wzajemnie prostopadłe, tj. $\mathbf v_i\cdot\mathbf v_j = 0$ dla $i\neq j$.
    \end{enumerate}

    Pierwszy warunek pozwala nam na znalezienie całej masy wektorów - przykładem mogą być $v_1=(2,1,0,0), v_2 = (0,1,1,0), v_3=(0,0,1,1)$. Jedynym problemem jest fakt, że nie są to wektory wzajemnie prostopadłe - ich wzajemne iloczyny skalarne nie są równe zeru. Z pomocą może przyjść \textbf{ortogonalizacja Grama-Schmitda} - algorytm przekształcania dowolnego zestawu wektorów liniowo niezależnych do bazy ortogonalnej.

    Dla zestawu trzech wektorów procedura przebiega następująco:

    \begin{align*}
        \mathbf u_1 & = \mathbf v_1,                                                                                                                                                              \\
        \mathbf u_2 & = \mathbf v_2 - \frac{\mathbf v_2\cdot \mathbf u_1}{\mathbf u_1\cdot \mathbf u_1}\mathbf u_1,                                                                               \\
        \mathbf u_3 & = \mathbf v_3 - \frac{\mathbf v_3\cdot \mathbf u_2}{\mathbf u_2\cdot \mathbf u_2}\mathbf u_2- \frac{\mathbf v_3\cdot \mathbf u_1}{\mathbf u_1\cdot \mathbf u_1}\mathbf u_1.
    \end{align*}
\end{proof}


\begin{zadanie}
    Wyznaczyć rzut ortogonalny wektora $\mathbf u=(1,-2,3)$ na prostą $L\subset \mathbb R^3$ o równaniach:

    $$x_1-x_3 = x_1+2x_2+x_3 = 0.$$

    Obliczyć odległość wektora $\mathbf u$ od tej prostej.
\end{zadanie}
\begin{proof}
    Algorytm postępowania:
    \begin{enumerate}
        \item Znaleźć jądro przekształcenia liniowego - to będzie nasza baza podprzestrzeni liniowej, w tym wypadku prostej.
        \item Przeprowadzić rzut ortogonalny wektora $\mathbf u$ na podprzestrzeń rozpinaną przez wektor $\mathbf v$ - będzie to opisane wzorem $P(\mathbf u) = \frac{\mathbf v\cdot \mathbf u}{\mathbf v\cdot \mathbf v}\mathbf v$.
        \item Aby obliczyć odległość wektora od prostej, należy od wejściowego wektora $u$ odjąć wektor uzyskany przez rzucenie na podprzestrzeń, a następnie wyliczyć jego długość.
    \end{enumerate}
\end{proof}

\begin{zadanie}
    Zbadać wypukłość zbioru $W\subset \mathbb R^2$ danego warunkami:

    $$x^4+y^2\leq 1 \and y\geq \max\{e^{x^2}-1, -\ln(x+1)\}\and x>-1.$$
\end{zadanie}

\begin{proof}
Przecięcie zbiorów wypukłych jest zbiorem wypukłym. Możemy rozbić powyższe warunki na następujące:

\begin{itemize}
\item $x^4+y^2\leq 1$
\item $y\geq e^{x^2}-1$ dla $x>0$
\item $y\geq -\ln(x+1)$ dla $x\leq0$.
\end{itemize}

(warunek $x>-1$ jest spełniany automatycznie przy konieczności pierwszego warunku).
\end{proof}

\begin{zadanie}
Wyznaczyć rozwiązanie równania różniczkowego:

$$y' - \frac yx = \frac{y^2}{\sqrt{x^2+1}}$$

spełniające warunek $y(1)=1$.
\end{zadanie}
\begin{proof}
Należy dokonać podstawienia:

$$y = vx,$$

rozwiązać równanie dla $v(x)$ a następnie podstawić $y$ z powrotem.
\end{proof}

\begin{zadanie}
Podać rozwiązanie ogólne równania różniczkowego:

$$y'' - 4y' + 4y = xe^{-x}.$$
\end{zadanie}

\begin{proof}
Należy wykonać podstawienie $z = y' - 2y$.
\end{proof}

\end{document}
